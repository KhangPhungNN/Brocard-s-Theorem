%LATEX by PNNK

\documentclass[12pt,a4paper]{article}
\usepackage[utf8]{vietnam}
\usepackage[left=1.5cm, right=1.5cm, top=1cm, bottom=2cm]{geometry}
\usepackage{graphicx}
\usepackage{mathtools}
\usepackage{amssymb}
\usepackage{amsthm}
\usepackage{nameref}
\usepackage{amsmath}
\usepackage{amsfonts}
\usepackage{enumitem}
\usepackage{indentfirst}
\usepackage{mathrsfs}


\usepackage{pgfplots}
\pgfplotsset{compat=1.15}
\usepackage{mathrsfs}
\usetikzlibrary{arrows}
\pagestyle{empty}

\definecolor{uuuuuu}{rgb}{0.26666666666666666,0.26666666666666666,0.26666666666666666}

\begin{document}
\begin{center}
	\textbf{\underline{{\Large Chứng minh định lý Brocard bằng kiến thức THCS}}}
\end{center}

\textbf{Phát biểu định lý:} Cho tứ giác toàn phần $ABCDXY$ nội tiếp đường tròn $(O);\space Z$ là giao điểm của $AC, DB$. Chứng minh rằng $O$ là trực tâm của tam giác $XYZ$.

\begin{tikzpicture}[line cap=round,line join=round,>=triangle 45,x=1cm,y=1cm]
	\clip(-5.75119863572696,-3.619600353577202) rectangle (13.560756218837444,7.321197365887557);
	\draw (-1.1198412799266677,4.465108187346908)-- (1.952704122239836,-0.5692211450811859);
	\draw (0.8920317919726709,3.5936935004108923)-- (-5.028756998508487,-0.5263733210017867);
	\draw (0.23062348648099065,6.189581262362677)-- (-1.53,0.76);
	\draw (-1.53,0.76)-- (10.626032037482737,-0.6224525850717072);
	\draw (10.626032037482737,-0.6224525850717072)-- (0.23062348648099065,6.189581262362677);
	\draw (0.23062348648099065,6.189581262362677)-- (1.952704122239836,-0.5692211450811859);
	\draw (0.23062348648099065,6.189581262362677)-- (-5.028756998508487,-0.5263733210017867);
	\draw (-1.1198412799266677,4.465108187346908)-- (10.626032037482737,-0.6224525850717072);
	\draw (-5.028756998508487,-0.5263733210017867)-- (10.626032037482737,-0.6224525850717072);
	\draw (-1.53,0.76) circle (3.727741522101188cm);
	\draw [dash pattern=on 4pt off 3pt] (1.4880253477553584,5.3656151818758655)-- (-1.53,0.76);
%	\draw (6.309949043193097,2.757546846132009) circle (5.482058789226331cm);
%	\draw (0.2221270985820833,4.805212206116102) circle (1.3843951287476999cm);
	\draw [dash pattern=on 4pt off 3pt] (0.23062348648099065,6.189581262362677)-- (-0.40145520442875066,0.6316555151223695);
	\draw [dash pattern=on 4pt off 3pt] (-0.7790603707396485,3.0758203735808247)-- (10.626032037482737,-0.6224525850717072);
	\draw  (1.4880253477553584,5.3656151818758655)-- (0.8920317919726709,3.5936935004108923);
	\draw  (1.4880253477553584,5.3656151818758655)-- (1.952704122239836,-0.5692211450811859);
	\draw  (1.4880253477553584,5.3656151818758655)-- (-1.1198412799266677,4.465108187346908);
	\begin{scriptsize}
		\draw [fill=uuuuuu] (-1.53,0.76) circle (2.5pt);
		\draw[color=uuuuuu] (-1.9076933607771975,0.6000009302585) node {\normalsize$O$};
		\draw [fill=uuuuuu] (0.8920317919726709,3.5936935004108923) circle (2.5pt);
		\draw[color=uuuuuu] (1.3419965551457308,3.659673767787805) node {\normalsize$B$};
		\draw [fill=uuuuuu] (-1.1198412799266677,4.465108187346908) circle (2.5pt);
		\draw[color=uuuuuu] (-1.3897384639156807,4.867565238360825) node {\normalsize$A$};
		\draw [fill=uuuuuu] (1.952704122239836,-0.5692211450811859) circle (2.5pt);
		\draw[color=uuuuuu] (1.9939246720187733,-1.0998540587311804) node {\normalsize$C$};
		\draw [fill=uuuuuu] (-5.028756998508487,-0.5263733210017867) circle (2.5pt);
		\draw[color=uuuuuu] (-5.463328307250099,-0.67989634292351013) node {\normalsize$D$};
		\draw [fill=uuuuuu] (10.626032037482737,-0.6224525850717072) circle (2pt);
		\draw[color=uuuuuu] (10.861038113452964,-0.96991748501967494) node {\normalsize$X$};
		\draw [fill=uuuuuu] (0.23062348648099065,6.189581262362677) circle (2pt);
		\draw[color=uuuuuu] (0.37408394247654203,6.563397511064584) node {\normalsize$Y$};
		\draw [fill=uuuuuu] (1.4880253477553584,5.3656151818758655) circle (2pt);
		\draw[color=uuuuuu] (1.7339570898995581,5.641479260503088) node {\normalsize$T$};
		\draw [fill=uuuuuu] (-0.14680675572277027,2.8708026180261657) circle (2pt);
		\draw[color=uuuuuu] (-0.058230323003365,2.2887667930109305) node {\normalsize$Z$};
		\draw [fill=uuuuuu] (-0.7790603707396485,3.0758203735808247) circle (2pt);
		\draw[color=uuuuuu] (-1.1497525586464566,3.227713233033979) node {\normalsize$W$};
		\draw [fill=uuuuuu] (-0.40145520442875066,0.6316555151223695) circle (2pt);
%		\draw[color=uuuuuu] (-0.35384276492342287,0.2860064034341317) node {\normalsize$I$};
	\end{scriptsize}
\end{tikzpicture}



\setlength{\parindent}{0pt}

\textbf{\underline{Solution.}}\\

Gọi $T$ là điểm Miquel của tứ giác toàn phần $ABCDXY$. Bằng cộng góc ta dễ chứng minh $T$ thuộc $XY.$\\

Ta có các tứ giác nội tiếp $AYTB, TBCX, YTCD$ từ đó có các biến đổi sau:

$\angle ATC=180^{\circ}-(\angle YTA+\angle XTC)=180^{\circ}-(2\angle ADC)=180^{\circ}-\angle AOC\Rightarrow$ Tứ giác $ATCO$ nội tiếp.
$\angle DTB=180^{\circ}-(\angle BTX+\angle DTY)=180^{\circ}-(2\angle BCD)=180^{\circ}-\angle BOD\Rightarrow$ Tứ giác $BTDO$ nội tiếp.\\

Ta có $OT$ là trục đẳng phương của $(ATCO)$ và $(TBOD)$.\\

Có: $\mathcal{P}_{Z/(TBOD)}=ZB.ZD=ZA.ZC=\mathcal{P}_{Z/(ATCO)}$ Suy ra $Z$ thuộc trục đẳng phương của $(ATCO)$ và $(TBOD)$. Do đó $O,Z,T$ thẳng hàng.\\

Tứ giác $ATCO$ nội tiếp có $OA=OC\Rightarrow TO$ là phân giác $\angle ATC\Rightarrow\angle ATO=\angle CTO$; có $\angle YTA=\angle XTC\Rightarrow\angle OTY=\angle OTX\Rightarrow ZT\perp XY. \qquad(*)$\\

Đường tròn ngoại tiếp các tam giác $AZB, DZC$ cắt nhau tại $W$.\\

Ta có: $\angle BWC=\angle BWZ+\angle CWZ=\angle ZAB+\angle ZDC=2\angle BDC=\angle BOC\Rightarrow$ Tứ giác $BWOC$ nội tiếp. Tương tự có tứ giác $AWOD$ nội tiếp.\\

Có: $\mathcal{P}_{Y/(AWOD)}=YA.YD=YB.YC=\mathcal{P}_{Y/(BWOC)}\Rightarrow\space Y,W,O$ thẳng hàng. Tương tự: $X,Z,W$ thẳng hàng.\\

Ta có $\angle ZWO=\angle CWZ+\angle CWO=\angle BDC+\angle OBC=\dfrac{(180^{\circ}-\angle BOC)}{2}+\dfrac{\angle BOC}{2}=90^{\circ}\newline\Rightarrow YW\perp OY.\enspace$ Kết hợp $(*)$ ta hoàn tất chứng minh.



\end{document}
